\chapter{Developer Documentation}\label{chapter:developer}

In this chapter it will be presented how the project is structured and what are the core principles and ideas behind it.

Some important notes:
\begin{enumerate}
    \item The entiry file is \lstinline{trajectory\_gen\_node.cpp} file 
    \item  Trajetory generator files are \lstinline{trajectory\_generator\_circular} (hpp and cpp) and \lstinline{trajectory\_generator\linear} (hpp and cpp)
    \item Project separates header and implementation, and also applies OOP principle such as polymorphism
\end{enumerate}


\section{Gazebo simulator}
In order to check the behaviour of the trajectory generator we have to see the result by our own eyes. To do so the docker environment was already preinstalled with  \href{https://gazebosim.org/home}{Gazebo simulator}.

To run it use this command in the opened terminal inside container
\begin{lstlisting}[language=bash]
roslaunch franka_gazebo panda.launch x:=-0.5 world:=$(rospack find demo_pkg)/world/pick_place.sdf controller:=cartesian_impedance_example_controller     rviz:=false
\end{lstlisting}

\section{Main program}
As mentioned before, main program is located in the \lstinline{trajectory\_gen\_node.cpp} and is connecting rosnode subscribers information to the trajectory classes as well as deciding on the next robots frame position.

Here is the full functionality of the main program 
\begin{enumerate}
    \item Subscribe to the "/franka\_state\_controller/franka\_states" to retrieve the data about the initial position and current force.
    \item Close gripper (grab the object) before starting main loop
    \item Assign the initial state of the program. There are three states which are running though the main loop
    \begin{enumerate}
        \item DOWN\_LINEAR state - apply linear trajectory until reaching a force threshold
        \item DOWN\_CIRCULAR state - once reaching destination (desk) start making circular movements to clean the board
        \item UP\_LINEAR state - go to the initial state specified by the program, generally a good practice, not to have force disturbance in the beginning from the previous actions on the robot, and finishing complying to the standard of "pick and place" operation.

    As a side note this implementation could be extended to so called "Behaviour tree", where each action can lead to multitude of action defined by the decision of the robots cognition.
    \end{enumerate}
    \item Once state UP\_LINEAR successfully transition into final state, finish the program
\end{enumerate}

\section{Trajectory generators and helper functions}

\subsection{VelocityProfile class}
Velocity profile is one of the simpler classes, however the most important one since it will be used by the both trajectory generators. 

Using knowledge from \ref{chapter:basis} this class takes as an input the following 
\begin{itemize}
    \item p\_start - Start point of movement
    \item p\_end - End point of movement
    \item q\_dot\_max - maximum speed the robot tries to reach 
    \item q\_double\_dot\_max - constant absolute value of acceleration 
\end{itemize}


The start point and the end point provide the total distance of the movement. The maximal velocity and the constant acceleration together determine the time for both acceleration and deceleration periods 
$$t_c = \frac{q_{dotmax}}{q_{doubledotmax}}$$ 

Since the total distance is already given, the distance for each period (acceleration, deceleration, or constant velocity) and the total movement time $t_f$ can be calculated 
$$t_f = 2t_c + (d_2 - q_{doubledot} * tc^2) / q_{dotmax} $$  

\subsection{Trajectory class}
Trajectory class is a small interface which provides decent enough abstraction for its child classes, so that they could be used together in sequential and similar task assignments. 

\begin{lstlisting}[language=bash]
class Trajectory {
    public:
        virtual ~Trajectory() {}
        virtual void update(double dt, double force = 0) = 0;
        virtual bool isEnded() = 0;
        virtual geometry_msgs::Point getPoint() = 0;
        virtual std::string getName() = 0;
    };
\end{lstlisting}

\subsection{Linear Trajectory class}

Linear Trajectory is a child class inherited from Trajectory thus applying its methods. Linear trajectory algorithms are heavily inspired by the \ref{chapter:basis} theoretical background, by using relative positions, norms for the direction, etc..

Using knowledge from \ref{chapter:basis} this class takes as an input the following 
\begin{itemize}
    \item c - triple vector from initial robot position to the end effector (chosen frame). Provide the notion of relative movement 
    \item normal\_vector\_hat - unit normal vector
    \item p\_start - starting point of movement
    \item p\_end - end point of movement
    \item q\_dot\_max - maximum speed of the robot
    \item q\_double\_dot\_max - constant absolute value of acceleration 
    \item force\_threshold - force which is acceptable, until the threshold is reached
\end{itemize}

The c vector points from the origin of the base to the origin of the new coordination system. The unit normal vector stands for the orientation of the plane of the new coordination system (in the same direction of z axis). Furthermore, cross product of the unit normal vector and the c vector (after normalization) gives the unit vector y, cross product of y and unit normal vector yields unit vector z. Subsequently, all these vectors x, y, z build the rotation matrix, which with the c vector are able to transform the coordinates in the new CoS back to the base coordination system. This method offers us applicability even to the coordinate transformations.

In addition, start point and end point (in the new CoS) define the direction of movement, and where the linear movement starts and ends, while maximal velocity, and constant acceleration serve for the inputs of velocity profile function, in order to create trapezoidal velocity profile. Force threshold should be set by the user, and once the detected force exceeds this threshold, the robot will stop. This is useful for real cases when robot touches the surface of the table and stops to draw a circle on it. The current position value (double) from the velocity profile function times unit vector in the moving direction plus start point should give the current position in vector form.

\subsection{Circular Trajectory class}

 Circular Trajectory is a child class inherited from Trajectory thus applying its methods. Circular trajectory algorithm is heavily inspired by the \ref{chapter:basis} theoretical background, by using relative positions to define the position according to the trigonometric rules of cosine and sine. 

Using knowledge from \ref{chapter:basis} this class takes as an input the following

\begin{itemize}
    \item c - triple vector from initial robot position to the end effector (chosen frame). Provide the notion of relative movement 
    \item normal\_vector\_hat - unit normal vector
    \item radius - radius of the circle
    \item q\_dot\_max - maximum speed of the robot
    \item q\_double\_dot\_max - constant absolute value of acceleration 
\end{itemize}

For the circular trajectory function, inputs are c vector, unit normal vector, radius(double), maximal velocity(double), and constant acceleration(double). The c vector and unit normal vector are the same as in the linear trajectory function, a rotation matrix is obtained based on them and coordinate transformations can thus be realized, which has already been described in detail previously. The radius means the circumference of the circle, which is also the total path needs to be travelled. The movement starts from the origin of the new CoS, it first moves to the (radius, 0, 0), to perform the circle drawing. Similarly, the last two inputs: maximal velocity and constant acceleration together with 0 (for start point) and circumference (for end point) will be input to the velocity profile function to generate trapezoidal velocity. Using the current position value acquired from the function, the ratio of the travelled path to the radius can also be gained. Futhermore, this ratio is of great importance, which gives the current position in vector form in the new CoS: (cos(ratio) * radius, radius * sin(ratio), 0). Finally, the movement should stop if the travelled path is larger than or equal to the circumference(2 * Pi * radius).